\\
Popularita a~používání průběžné integrace spolu s~automatizovanou kontrolou kódu v~moderních projektech s~otevřeným zdrojovým kódem se neustále zvyšuje. Tato bakalářská práce se snaží o~vysvětlení základů a~poukázání na praktické příklady. Implementační část práce obsahuje popis navrhnutých a naimplementovaných vylepšení do ManageIQ bota, které byly založené na základě informací jako například: osvědčené postupy, rozhovory s~vývojáři bota, atd. Tato bakalářská práce byla vypracována ve~spolupráci s~firmou Red Hat, Inc.\\[1em]
\indent Řešení této práce tvoří části jako průběžná integrace, automatizovaná kontrola kódu a~implementace s příslušným detailním popisem. Průběžná integrace sehrává klíčovou roli v procesu softwarového vývoje. Požadavky této praktiky jsou využívání verzovacího systému, unit testů a jejich automatizování, mechanizmus na zpětnou vazbu a build skript. Tyto nutné požadavky tvoří základ celého procesu, ve~kterém skupina vývojářů jednoduše a rychle integruje změnu do plně fungujícího produktu. Automatizace tohoto procesu se nazývá integration build, což značí průběh dostavení změny od jejího vytvoření až k jejímu plnému nasazení do softwarového produktu. V průběhu procesu nasazování vytvořené změny dochází ke~kontrole dané změny mnoha nástroji a následně i testy. K přerušení procesu může dojít v mnoha případech, při kterých je nutné informovat vývojáře ne jen o aktuálním průběhu procesu, ale i o tom, co způsobilo přerušení. Kvůli tomuto se vyžaduje mechanizmus na zpětnou vazbu, která je dostupná pro všechny vývojáře. Na základě možných přerušení procesu byla navrhnuta různá doporučení, jak se jim vyhnout co nejvíce efektivním způsobem.\\[1em]
\indent Automatizovaná kontrola kódu tvoří důležitou součást průběžné integrace, která odhaluje množství chyb způsobených nepozorností vývojářů. Z toho důvodu, že automatizovaná kontrola kódu odhalí množství chyb, ale zdaleka ne všechny, je brán důležitý zřetel na manuální kontrolu kódu. Kontrola kódu se dá automatizovat na základě pravidel, které definují chyby, čímž dochází k omezení kontroly na základě konečné množiny pravidel. Tyto fakty svědčí o tom, že manuální a automatizovaná kontrola kódu se navzájem doplňují a obě jsou součástí průbežné integrace.\\[1em]
\indent Implementaci této práce tvoří návrh a vylepšení ManageIQ bota. Na základě obsažených nedostatků bota byla navrhnuta a následně naimplementována tato vylepšení. Vylepšení jsou tvořena: chybějícími příkazy bota, integrací služby Gitter za účelem upozornění vývojářů, integrací aplikačního programovacího rozhraní GitHub Status, přidaním odstraňování zpráv o neslučitelném stavu pull requestu, dále integrací Pronta a vytvořením nového formátovacího způsobu pro bota. Součastí řešení je i vytvoření unit testů k dané implementaci. Tato vylepšení byla po jejich dokončení přidána do bota pomocí pull requestů, ze kterých byla některá ihned přidána a zbývající ponechána ve stádiu kontroly kódu.\\[1em]
\indent Cílem této práce bylo seznámit čtenáře se~základy průběžné integrace a automatizovanou kontrolou kódu. Na základě těchto informací se podařilo navrhnout a implementovat spolu s unit testy nové vylepšení do ManageIQ bota. Následně tyto výsledky implementace byly prodiskutovány s vývojáři, kteří jsou zodpovědní za jeho údržbu.
