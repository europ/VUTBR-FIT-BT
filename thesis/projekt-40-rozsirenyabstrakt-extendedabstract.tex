\\
\textbf{Úvod}\\[0.25em]
Popularita a~používanosť priebežnej integrácie spolu s~automatizovanou kontrolou kódu v~moderných projektoch s~otvoreným zdrojovým kódom sa neustále zvyšuje. Táto bakalárska práca sa snaží o~vysvetlenie základov a~poukázať na praktické príklady týchto praktík. Implementačná časť práce obsahuje popis navrhnutých a naimplementovaných vylepšení do ManageIQ bota ktoré boli založené na základe informácii ako napríklad: osvedčené postupy týchto praktík, rozhovor s~vývojármi tohto bota, atď. Táto bakalárska práca bola vypracovaná v~spolupráci s~firmou Red Hat, Inc.\\[1em]
\textbf{Řešení}\\[0.25em]
Riešením tejto práce tvoria časti ako priebežná integrácia, automatizovaná kontrola kódu a~implementácia s príslušným detailným popisom.\\[0.25em]
\indent Priebežná integrácia odohráva kľúčovú rolu v procese softvérového vývoja. Požiadavkami tejto praktiky sú využívanie verzovacieho systému, unit testov a ich automatizovanie, mechanizmus na spätnú väzbu a build skript. Tieto nutné požiadavky tvoria základ celého procesu v ktorom skupina vývojárov jednoducho a rýchlo integruje zmenu do plne fungujúceho produktu. Automatizácia tohto procesu sa nazýva integration build čo značí priebeh dostavenia zmeny od jej vytvorenia až k jej plnému nasadeniu do softvérového produktu. V priebehu tohto procesu nasaďovania vytvorenej zmeny dochádza ku kontrole danej zmeny mnohými nástrojmi a následne aj testami. K prerušeniu tohto procesu môže dôjsť v mnohých prípadoch pri ktorých je nutné informovať vývojára nie len o aktuálnom priebehu procesu ale aj o tom čo spôsobilo prerušenie. Kvôli tomuto sa vyžaduje mechanizmus na spätnú väzbu ktorý je dostupná pre všetkých vývojárov. Na základe možných prerušení procesu boli navrhnuté rôzne odporučenia ako sa im vyhnúť čím efektívnejším spôsobom.\\[0.25em]
\indent Automatizovaná kontrola kódu tvorí dôležitú súčasť priebežnej integrácie ktorá odhaľuje množstvo chýb spáchané nepozornosťou vývojárov. Z dôvodu toho, že automatizovaná kontrola kódu odhalí množstvo chýb ale nie všetky, je braný dôležitý zreteľ na manuálnu kontrolu kódu. Kontrola kódu sa dá automatizovať na základe pravidiel ktoré definujú chyby čím dochádza k limitácii kontrole na základe konečnej množine pravidiel. Tieto fakty svedčia o tom že manuálna a automatizovaná kontrola kódu sa navzájom dopĺňajú a obe sú súčasťou priebežnej integrácie.\\[0.25em]
\indent Implementáciu tejto práce tvorí návrh a vylepšenie ManageIQ bota. Na základe obsahovaných nedostatkov bota boli navrhnuté a následne naimplementované tieto vylepšenia. Vylepšenia tvorili: chýbajúce príkazy bota, integrácia služby Gitter za účelom upozornenia vývojárov, integrácia aplikačného programovacieho rozhrania GitHub Status, pridanie odtranovania správ o nezlučiteľnom stave pull requestu, integrácia Pronta a vytvorenie nového formátovacieho spôsobu pre bota. Súčasť implementácie tvorí aj vytvorenie unit testov k danej implementácii. Tieto vylepšenia boli po ich dokončení pridané do bota pomocou pull requestov z ktorých boli niektoré ihneď pridané a zvyšné sú ešte v štádiu kontroly kódu.\\[0.25em]
\textbf{Shrnutí}\\[0.25em]
Cieľom tejto práce bolo oboznámiť čitaľa so záklamdmi priebežnej integrácie a automatizovanej kontrole kódu. Na základe týchto informácii sa podarilo návrhnúť a implementovať spolu s unit testami nové vylepšenia do ManageIQ bota. Následne tieto výsledky implementácie boli prediskutované s vývojármi ktorý sú zodpovedný za udržiavania tohto bota.
